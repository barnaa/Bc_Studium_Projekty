% ITY: Projekt 2, Andrej Barna (xbarna01), 2014/2015
\documentclass[11pt,a4paper,twocolumn]{article}
\usepackage[left=1.5cm,text={18cm, 25cm},top=2.5cm]{geometry}
\usepackage{times}
\usepackage[czech]{babel}
\usepackage[utf8]{inputenc}
\usepackage[T1]{fontenc}
\usepackage{amsmath}
\usepackage{amsthm}
\usepackage{amsfonts}


\begin{document}
\begin{titlepage}
\begin{center}
\Huge
\textsc{Fakulta informačních technologií\\Vysoké učení technické v~Brně}
\vspace{\stretch{0.382}}
\LARGE \\Typografie a publikování\,--\,2. projekt\\
Sazba dokumentů a~matematických výrazů\\
\vspace{\stretch{0.618}}
\end{center}
\Large 2015 \hfill Andrej Barna
\end{titlepage}
\theoremstyle{definition}
\newtheorem{dfn}{Definice}[section]
\theoremstyle{plain}
\newtheorem{alg}[dfn]{Algoritmus}
\newtheorem{vet}{Věta}
\section*{Úvod}
V~této úloze si vyzkoušíme sazbu titulní strany, matematických vzorců, prostředí a dalších textových struktur obvyklých pro technicky zaměřené texty (například rovnice (\ref{eq:eq1}) nebo definice \ref{def:d1-1} na straně \pageref{def:d1-1}).

Na titulní straně je využito sázení nadpisu podle optického středu s~využitím zlatého řezu. Tento postup byl probírán na přednášce.


\section{Matematický text}
Nejprve se podíváme na sázení matematických symbolů a výrazů v~plynulém textu. Pro množinu $V$ označuje $\mathrm{card(\mathnormal{V})}$ kardinalitu $V$.
Pro množinu $V$ reprezentuje $V^*$ volný monoid generovaný množinou $V$ s~operací konkatenace.
Prvek identity ve volném monoidu $V^*$ značíme symbolem $\varepsilon$.
Nechť$V^+=V^*-\{\varepsilon\}$. Algebraicky je tedy $V^+$ volná pologrupa generovaná množinou $V$ s~operací konkatenace.
Konečnou neprázdnou množinu $V$ nazvěme abeceda.
Pro $w\in V^*$ označuje $|w|$ délku řetězce $w$. Pro $W\subseteq V$ označuje occur($w, W$) počet výskytů symbolů z~$W$ v~řetězci $w$ a sym($w,i$) určuje $i$-tý symbol řetězce $w$; například $\text{sym}(abcd,3)=c$.

%Niekde boli doplnane zaporne medzery, aby nevznikali velke medzery v niektorych vyrazoch, najma tam, kde boli zatvorky
Nyní zkusíme sazbu definic a vět s~využitím balíku \texttt{amsthm}.
\begin{dfn}\label{def:d1-1} \emph{Bezkontextová gramatika} je čtveřice $G=(V,T,P,S)$, kde $V$ je totální abeceda, $T\!\subseteq\!V$ je abeceda terminálů, $S\!\in\!(V\!-\!T)$ je startující symbol a $P$ je konečná množina \emph{pravidel} tvaru $q\colon A\to\alpha$, kde $A\!\in\!(V\!-\!T)$, $\alpha \in V^*$ a $q$ je návěští tohoto pravidla. Nechť $N=V\!-\!T$ značí abecedu neterminálů. Pokud $q\colon A\to\alpha\in P,\gamma,\delta\in V^*,G$ provádí derivační krok z~$\gamma A\delta$ do $\gamma\alpha\delta$ podle pravidla $q\colon A\to\alpha$, symbolicky píšeme $\gamma A\delta \Rightarrow \gamma \alpha \delta[q\colon A\to\alpha]$ nebo zjednodušeně $\gamma A\delta\Rightarrow\gamma\alpha\delta$. Standardním způsobem definujeme $\Rightarrow^m$, kde $m\geq 0$. Dále definujeme tranzitivní uzávěr $\Rightarrow^+$ a tranzitivně-reflexivní uzávěr $\Rightarrow^*$.\end{dfn}

Algoritmus můžeme uvádět podobně jako definice textově, nebo využít pseudokódu vysázeného ve vhodném prostředí (například \texttt{algorithm2e}).

\begin{alg} Algoritmus pro ověření bezkontextovosti gramatiky. Mějme gramatiku $G = (N, T, P, S)$.
 \begin{enumerate}
 \item\label{al:1-2:it-1} Pro každé pravidlo $p\in P$ proveď test, zda $p$ na levé straně obsahuje právě jeden symbol z~$N$.
 \item Pokud všechna pravidla splňují podmínku z~kroku \ref{al:1-2:it-1}, tak je gramatika $G$ bezkontextová.
 \end{enumerate}\end{alg}

\begin{dfn} \emph{Jazyk} definovaný gramatikou $G$ definujeme jako $L(G)=\{w\in T^*|S\Rightarrow^*w\}$.\end{dfn}

\subsection{Podsekce obsahující větu}

\begin{dfn} Nechť $L$ je libovolný jazyk. $L$ je \emph{bezkontextový jazyk}, když a jen když $L=L(G)$, kde $G$ je libovolná bezkontextová gramatika.\end{dfn}

\begin{dfn} Množinu $\mathcal{L}_{CF}=\{L|L $ je bez\-kon\-tex\-to\-vý jazyk\} nazýváme \emph{třídou bezkontextových jazyků}.\end{dfn}

\begin{vet}\label{vet:1} Nechť $L_{abc}=\{a^nb^nc^n|n\geq 0\}$. Platí, že $L_{abc}\notin\mathcal{L}_{CF}$.\end{vet}

\begin{proof} Důkaz se provede pomocí Pumping lemma pro bezkontextové jazyky, kdy ukážeme, že není možné, aby platilo, což bude implikovat pravdivost věty \ref{vet:1}.\end{proof}

\section{Rovnice a odkazy}

Složitější matematické formulace sázíme mimo plynulý text. Lze umístit několik výrazů na jeden řádek, ale pak je třeba tyto vhodně oddělit, například příkazem \verb|\quad|.

$$\sqrt[x^2]{y_0^3} \quad \mathbb{N}=\{0,1,2,\ldots\} \quad x^{y^y}\not = x^{yy} \quad z_{i_j}\not\equiv z_{ij}$$

V~rovnici (\ref{eq:eq1}) jsou využity tři typy závorek s~různou explicitně definovanou velikostí.

\begin{eqnarray}
\label{eq:eq1}\left\{\Bigl[(a+b)*c\Bigr]^d+1\right\} &=& x\\
\lim\limits _{x\rightarrow\infty}\frac{\sin^2x+\cos^2x}{4} &=& y\nonumber
\end{eqnarray}

V~této větě vidíme, jak vypadá implicitní vysázení limity $\lim_{n\rightarrow\infty}f(n)$ v~normálním odstavci textu. Podobně je to i s~dalšími symboly jako $\sum_1^n$ či $\bigcup_{A\in\mathcal{B}}$. V~případě vzorce $\lim\limits_{x\rightarrow 0}\frac{\sin x}{x} \!=\! 1 $ jsme si vynutili méně úspornou sazbu příkazem \verb|\limits|.

\begin{eqnarray}
\int\limits _a^bf(x)\mathrm\,{d}x & = & -\int_b^af(x)\,\mathrm{d}x\\
\left(\sqrt[5]{x^4}\right)'=\left(x^{\frac{4}{5}}\right)'&=&\frac{4}{5}x^{-\frac{1}{5}}=\frac{4}{5\sqrt[5]{x}}\\
\overline{\overline{A\vee B}} &=& \overline{\overline{A}\wedge\overline{B}}
\end{eqnarray}

\section{Matice}

Pro sázení matic se velmi často používá prostředí \texttt{array} a závorky (\verb|\left|, \verb|\right|). 

%Pouzil som overrightarrow, pretoze na serveri Merlin som pri preklade dostaval SIGSEGV error
$$\left(\begin{array}{c c}
a+b & b-a \\
\widehat{\xi+\omega} & \hat{\pi}\\
\overrightarrow{a} & \overleftrightarrow{AC}\\
0 & \beta
\end{array}\right)$$

$$\mathbf{A}=\left|\left|\begin{array}{c c c c}
a_{11}&a_{12}&\ldots&a_{1n}\\
a_{21}&a_{22}&\ldots&a_{2n}\\
\vdots&\vdots&\ddots&\vdots\\
a_{m1}&a_{m2}&\ldots&a_{mn}
\end{array}\right|\right|$$

$$\left|\begin{array}{c c}
t&u\\
v&w
\end{array}\right|=tw-uv$$

Prostředí \texttt{array} lze úspěšně využít i jinde.

$$\binom{n}{k}=\left\{\begin{array}{l l}\frac{n!}{k!(n-k)!}&\text{pro }0\leq k\leq n\\0&\text{pro }k<0\text{ nebo }k>n\end{array}\right.$$

\section{Závěrem}

V~případě, že budete potřebovat vyjádřit matematickou konstrukci nebo symbol a nebude se Vám dařit jej nalézt v~samotném \LaTeX u, doporučuji prostudovat možnosti balíku maker \AmS -\LaTeX.
Analogická poučka platí obecně pro jakoukoli konstrukci v~\TeX u.

\end{document}
